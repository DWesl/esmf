% $Id$

\begin{enumerate}

% See https://github.com/esmf-org/esmf/issues/131
\item {\bf I/O of NetCDF files requires a real MPI library.}
Currently I/O of NetCDF files (which uses PIO) requires a real MPI
library: it cannot be done with ESMF\_COMM set to "mpiuni".

\item {\bf Limited data formats supported.}
Currently a small fraction of the anticipated data formats is implemented by 
ESMF.  The data I/O uses NetCDF format, and ESMF Info
I/O uses JSON format.  Different libraries are employed for these
different formats.  In future development, a more centralized I/O technique
will likely be defined to provide efficient utilities with a set of standard
APIs that will allow manipulation of multiple standard formats.  Also, the 
ability to automatically detect file formats at runtime will be developed.

\item {\bf Some limitations with multi-tile I/O.}
There are a few limitations when doing I/O on multi-tile Arrays and
Fields (e.g., a cubed sphere grid represented as a six-tile grid): This
I/O requires at least as many PETs as there are tiles, and for I/O of
ArrayBundles and FieldBundles, all Arrays / Fields in the bundle must
contain the same number of tiles.

% See https://github.com/esmf-org/esmf/issues/184 for more details on
% these issues with I/O of replicated dimensions.
\item {\bf Replicated dimensions.}
I/O of Arrays / Fields with replicated dimensions (section
\ref{sec:array:usage:replicated_dims}) is only partially working. In
most situations, replicated dimensions appear as dimensions in the
output file; ideally, these replicated dimensions would be removed in
the output file, and we plan to make that change in the future.
Furthermore, slices of the replicated dimensions other than the first
can have garbage values in the output file. In addition, there is an
inconsistency when outputting Arrays / Fields that have a decomposition
with more than one DE per PET: in this case, replicated dimensions are
removed in the output file. Finally, I/O cannot be performed on
multi-tile Arrays / Fields with replicated dimensions.

\end{enumerate}
