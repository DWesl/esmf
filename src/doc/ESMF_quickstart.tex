\section{Quick Start}
\label{sec:QuickStart}

This section gives a brief description of how to get the ESMF software, build 
it, and run the self-tests to verify the installation was successful. There is 
also a short guide for using the bundled ESMF command line tools. More detailed 
information on each of these steps is provided in sections \ref{sec:TechOver},
 \ref{sec:TechOver2} and \ref{sec:CLTs}, respectively.

With a growing user community requiring access to ESMF, central computing 
resources have started providing system wide ESMF installations. The availablity 
of center-managed ESMF installations dramatically increases the ease of use of 
ESMF. Practically it means that if you are working on a system (such as 
{\it Jaguar}) that offers a standard ESMF installation, you do not have to 
download, build and validate your own ESMF installation from source! Instead you 
can proceed directly to using ESMF as a programming library or through access to 
the bundled command line tools as described in sections \ref{sec:Use} and 
\ref{sec:CLTs}, respectively.

\subsection{Downloading ESMF}
\label{sec:download}
ESMF is distributed via releases on GitHub. Each release page contains release notes,
known issues, and links to supported platforms, documentation, and other related information
Releases on GitHub can be found from the ESMF web page via:
\begin{verbatim}
    http://www.earthsystemmodeling.org -> Download
\end{verbatim}

\subsection{Directory Structure}
The current list of directories includes the following:
\begin{itemize}
\item README
\item build
\item build\_config
\item makefile
\item scripts
\item src
\end{itemize}

The {\tt build\_config} directory contains subdirectories for
different operating system and compiler combinations. This is
a useful area to examine if porting ESMF to a new platform.

\subsection{Building ESMF}

After downloading and unpacking the ESMF tar file, the build procedure is:
\begin{enumerate}
\item Set the required environment variables. 
\item Type {\tt make info} to view and verify your settings
\item Type {\tt make } to build the library.
\item Type {\tt make check } to run self-tests to verify
the build was successful.
\end{enumerate}
See the following subsections for more information on each of these steps. Also
consult section \ref{sec:TechOver} for a complete discussion of the the ESMF
build process.

\subsubsection{Environment variables}

The syntax for setting environment variables depends on which shell
you are running.  Examples of the two most common ways to set 
an environment variable are:
\begin{description}
\item[ksh] {\tt  export ESMF\_DIR=/home/joeuser/esmf}
\item[csh] {\tt  setenv ESMF\_DIR /home/joeuser/esmf}
\end{description}

The shell environment variables listed below are the ones most
frequently used.  There are others which address needs on specific
platforms or are needed under more unusual circumstances;
see section \ref{sec:TechOver} for the full list.
\begin{description}

\item[ESMF\_DIR]
The environment variable {\tt ESMF\_DIR} must be set to the full pathname 
of the top level ESMF directory before building the framework.  This is the 
only environment variable which is required to be set on all platforms under 
all conditions.

\item[ESMF\_BOPT]
This environment variable controls the build option. To make a debuggable
version of the library set {\tt ESMF\_BOPT} to {\tt g} before building. The
default is {\tt O} (capital oh) which builds an optimized version of the 
library. If {\tt ESMF\_BOPT} is {\tt O}, {\tt ESMF\_OPTLEVEL} can also be set
to a numeric value between 0 and 4 to select a specific optimization level.

\item[ESMF\_COMM]
On systems with a vendor-supplied MPI communications library, the vendor library 
is chosen by default for communications. On these systems {\tt ESMF\_COMM} is
set to {\tt mpi}, signaling to the ESMF build system to use the vendor MPI
implementation.
For other systems (e.g. Linux or Darwin) where a multitude of MPI
implementations are available, {\tt ESMF\_COMM} must be set to indicate which
implementation is used to build the ESMF library. Set {\tt ESMF\_COMM} according
to your situation to: {\tt mpt, mpich, mpich1, mpich2, mpich3, mvapich2, lam, openmpi}
or {\tt intelmpi}. {\tt ESMF\_COMM} may also be set to {\tt user} indicating 
that the user will set all the required flags using advanced ESMF environment
variables.  Some individual MPI builds may create additional libraries that 
need to be linked in, such as the legacy C++ bindings. These may be specified 
via the {\tt ESMF\_CXXLINKLIBS} and {\tt ESMF\_F90LINKLIBS} environment
variables.

Alternatively, ESMF comes with a single-processor MPI-bypass library which is
the default for Linux and Darwin systems. To force the use of this bypass
library set {\tt ESMF\_COMM} equal to {\tt mpiuni}.

\item[ESMF\_COMPILER]
The ESMF library build requires a working Fortran90 and C++ compiler. On 
platforms that don't come with a single vendor supplied compiler suite
(e.g. Linux or Darwin) {\tt ESMF\_COMPILER} must be set to select which Fortran
and C++ compilers are being used to build the ESMF library. Notice that setting
the {\tt ESMF\_COMPILER} variable does {\em not} affect how the compiler
executables are located on the system. {\tt ESMF\_COMPILER} (together with
{\tt ESMF\_COMM}) affect the name that is expected for the compiler executables.
Furthermore, the {\tt ESMF\_COMPILER} setting is used to select compiler and
linker flags consistent with the compilers indicated.

By default Fortran and C++ compiler executables are expected to be located in
a location contained in the user's {\tt PATH} environment variable. This means
that if you cannot locate the correct compiler executable via the {\tt which}
command on the shell prompt the ESMF build system won't find it either!

There are advanced ESMF environment variables that can be used to select 
specific compiler executables by specifying the full path. This can be used to
pick specific compiler executables without having to modify the {\tt PATH}
environment variable.

Use 'make info' to see which compiler executables the ESMF build system will
be using according to your environment variable settings.

To see possible values for {\tt ESMF\_COMPILER}, cd to 
{\tt \$ESMF\_DIR/build\_config} and list the directories there. The first part 
of each directory name corresponds to the output of 'uname -s' for this 
platform. The second part contains possible values for {\tt ESMF\_COMPILER}. In
some cases multiple combinations of Fortran and C++ compilers are possible, e.g.
there is {\tt intel} and {\tt intelgcc} available for Linux. Setting 
{\tt ESMF\_COMPILER} to {\tt intel} indicates that both Intel Fortran and 
C++ compilers are used, whereas {\tt intelgcc} indicates that the Intel Fortran
compiler is used in combination with GCC's C++ compiler.

If you do not find a configuration that matches your situation you will need to
port ESMF.

\item[ESMF\_ABI]
If a system supports 32-bit and 64-bit (pointer wordsize) application binary
interfaces (ABIs), this variable can be set to select which ABI to use. Valid 
values are {\tt 32} or {\tt 64}. By default the most common ABI is chosen. On
x86\_64 architectures three additional, more specific ABI settings are available,
{\tt x86\_64\_32}, {\tt x86\_64\_small} and {\tt x86\_64\_medium}.

\item[ESMF\_SITE]
Build configure file site name or the value default. If not set, then the value
of default is assumed. When including platform-specific files, this value is
used as the third part of the directory name (parts 1 and 2 are the
ESMF\_OS value and ESMF\_COMPILER value, respectively.)

\item[ESMF\_ETCDIR]
If a user wants to add Attribute package specification files for their own 
customized Attribute packages, this is where they should go.  ESMF will look in 
this directory for files that specify which Attributes are in an Attribute 
package for certain ESMF objects, and what the appropriate initial values would 
be for those Attributes.  The format for these Attribute package specification 
files is to be defined in a future ESMF release.  This environment variable is 
largely for internal use at this point. 

\item[ESMF\_INSTALL\_PREFIX]
This variable specifies the prefix of the installation path used during the
installation process accessible thought the install target. Libraries, F90
module files, header files and documentation all are installed relative to
{\tt ESMF\_INSTALL\_PREFIX} by default. The {\tt ESMF\_INSTALL\_PREFIX} may be
provided as absolute path or relative to {\tt ESMF\_DIR}.

\end{description}


\subsubsection{GNU make}
\input{../../build/doc/user_make}

\subsubsection{make info}
{\tt make info} is a command that assists the user in verifying that the ESMF 
variables have been set appropriately. It also tells the user the paths to 
various libraries e.g. MPI that are set on the system. The user to review 
this information to verify their settings. In the case of a build failure, 
this information is invaluable and will be the first thing asked for by the
ESMF support team. Below is an {\bf example output} from {\tt make info}: 
 
\begin{verbatim}

--------------------------------------------------------------
Make version:
GNU Make 3.80
Copyright (C) 2002  Free Software Foundation, Inc.
This is free software; see the source for copying conditions.
There is NO warranty; not even for MERCHANTABILITY or FITNESS FOR A
PARTICULAR PURPOSE.

--------------------------------------------------------------
Fortran Compiler version:
Intel(R) Fortran Compiler for applications running on Intel(R) 64, \
	Version 10.1    
Build 20081024 Package ID: l_fc_p_10.1.021
Copyright (C) 1985-2008 Intel Corporation.  All rights reserved.

Version 10.1 

--------------------------------------------------------------
C++ Compiler version:
Intel(R) C++ Compiler for applications running on Intel(R) 64, Version 10.1    
Build 20081024 Package ID: l_cc_p_10.1.021
Copyright (C) 1985-2008 Intel Corporation.  All rights reserved.

Version 10.1 

--------------------------------------------------------------
Preprocessor version:
gcc (GCC) 4.1.2 20070115 (SUSE Linux)
Copyright (C) 2006 Free Software Foundation, Inc.
This is free software; see the source for copying conditions.  There is NO
warranty; not even for MERCHANTABILITY or FITNESS FOR A PARTICULAR PURPOSE.


--------------------------------------------------------------
ESMF_VERSION_STRING:    5.1.0
--------------------------------------------------------------
 
--------------------------------------------------------------
 * User set ESMF environment variables *
ESMF_OS=Linux
ESMF_TESTMPMD=ON
ESMF_TESTHARNESS_ARRAY=RUN_ESMF_TestHarnessArrayUNI_2
ESMF_DIR=/nobackupp10/scvasque/daily_builds/intel/esmf
ESMF_TESTHARNESS_FIELD=RUN_ESMF_TestHarnessFieldUNI_1
ESMF_TESTWITHTHREADS=OFF
ESMF_COMM=mpiuni
ESMF_INSTALL_PREFIX= /nobackupp10/scvasque/daily_builds/intel/esmf/.. \
	/install_dir
ESMF_TESTEXHAUSTIVE=ON
ESMF_BOPT=g
ESMF_SITE=default
ESMF_ABI=64
ESMF_COMPILER=intel
 
--------------------------------------------------------------
 * ESMF environment variables *
ESMF_DIR: /nobackupp10/scvasque/daily_builds/intel/esmf
ESMF_OS:                Linux
ESMF_MACHINE:           x86_64
ESMF_ABI:               64
ESMF_COMPILER:          intel
ESMF_BOPT:              g
ESMF_COMM:              mpiuni
ESMF_SITE:              default
ESMF_PTHREADS:          ON
ESMF_OPENMP:            ON
ESMF_ARRAY_LITE:        FALSE
ESMF_NO_INTEGER_1_BYTE: FALSE
ESMF_NO_INTEGER_2_BYTE: FALSE
ESMF_FORTRANSYMBOLS:    default
ESMF_DEFER_LIB_BUILD:   ON
ESMF_TESTEXHAUSTIVE:    ON
ESMF_TESTWITHTHREADS:   OFF
ESMF_TESTMPMD:          ON
ESMF_TESTSHAREDOBJ:     OFF
ESMF_TESTFORCEOPENMP:   OFF
ESMF_TESTHARNESS_ARRAY: RUN_ESMF_TestHarnessArrayUNI_2
ESMF_TESTHARNESS_FIELD: RUN_ESMF_TestHarnessFieldUNI_1
ESMF_MPIRUN:            /nobackupp10/scvasque/daily_builds/intel/esmf/src/ \
                         Infrastructure/stubs/mpiuni/mpirun
 
--------------------------------------------------------------
 * ESMF environment variables pointing to 3rd party software *
 
--------------------------------------------------------------
 * ESMF environment variables for final installation *
ESMF_INSTALL_PREFIX:    /nobackupp10/scvasque/daily_builds/intel/esmf/../ \
	install_dir
ESMF_INSTALL_HEADERDIR: include
ESMF_INSTALL_MODDIR:    mod/modg/Linux.intel.64.mpiuni.default
ESMF_INSTALL_LIBDIR:    lib/libg/Linux.intel.64.mpiuni.default
ESMF_INSTALL_BINDIR:    bin/bing/Linux.intel.64.mpiuni.default
ESMF_INSTALL_DOCDIR:    doc
 
 
--------------------------------------------------------------
 * Compilers, Linkers, Flags, and Libraries *
Location of the preprocessor:      /usr/bin/gcc
Location of the Fortran compiler:  /nasa/intel/fce/10.1.021/bin/ifort
Location of the Fortran linker:    /nasa/intel/fce/10.1.021/bin/ifort
Location of the C++ compiler:      /nasa/intel/cce/10.1.021/bin/icpc
Location of the C++ linker:        /nasa/intel/cce/10.1.021/bin/icpc

Fortran compiler flags:
ESMF_F90COMPILEOPTS: -g -fPIC -m64 -mcmodel=small -threads  -openmp
ESMF_F90COMPILEPATHS: -I/nobackupp10/scvasque/daily_builds/intel/esmf/mod/ \
	modg/Linux.intel.64.mpiuni.default -I/nobackupp10/scvasque/daily_builds \
	/intel/esmf/src/include 
ESMF_F90COMPILECPPFLAGS: -DESMF_TESTEXHAUSTIVE -DSx86_64_small=1 \
	-DESMF_OS_Linux=1  -DESMF_MPIUNI
ESMF_F90COMPILEFREECPP: 
ESMF_F90COMPILEFREENOCPP: 
ESMF_F90COMPILEFIXCPP: 
ESMF_F90COMPILEFIXNOCPP: 

Fortran linker flags:
ESMF_F90LINKOPTS:  -m64 -mcmodel=small -threads  -openmp
ESMF_F90LINKPATHS: -L/nobackupp10/scvasque/daily_builds/intel/esmf/lib/libg/ \
	Linux.intel.64.mpiuni.default  -L/nasa/sgi/mpt/1.25/lib -L/nasa/intel/ \
	cce/10.1.021/lib/shared -L/nasa/intel/fce/10.1.021/lib/shared -L/nasa/ \
	intel/cce/10.1.021/lib -L/nasa/intel/fce/10.1.021/lib -L/nasa/intel/cce/ \
	10.1.021/lib -L/usr/lib64/gcc/x86_64-suse-linux/4.1.2/ -L/usr/lib64/gcc/ \
	x86_64-suse-linux/4.1.2/../../../../lib64
ESMF_F90LINKRPATHS: 
	-Wl,-rpath,/nobackupp10/scvasque/daily_builds/intel/esmf/lib/libg/ \
	Linux.intel.64.mpiuni.default
ESMF_F90LINKLIBS:  -limf -lsvml -lm -lipgo -lguide -lstdc++ -lirc -lgcc_s \
	-lgcc -lirc -lpthread -lgcc_s -lgcc -lirc_s -ldl -lrt -ldl
ESMF_F90ESMFLINKLIBS: -lesmf  -limf -lsvml -lm -lipgo -lguide -lstdc++ -lirc \
	-lgcc_s -lgcc -lirc -lpthread -lgcc_s -lgcc -lirc_s -ldl -lrt -ldl

C++ compiler flags:
ESMF_CXXCOMPILEOPTS: -g -fPIC -m64 -mcmodel=small -pthread  -openmp
ESMF_CXXCOMPILEPATHS: -I/nobackupp10/scvasque/daily_builds/intel/ esmf/src/ \
	include  -I/nobackupp10/scvasque/daily_builds/intel/esmf/src/Infrastructure \
	/stubs/mpiuni
ESMF_CXXCOMPILECPPFLAGS: -DESMF_TESTEXHAUSTIVE  -DSx86_64_small=1 \
	-DESMF_OS_Linux=1 -D__SDIR__='' -DESMF_MPIUNI

C++ linker flags:
ESMF_CXXLINKOPTS:  -m64 -mcmodel=small -pthread  -openmp
ESMF_CXXLINKPATHS: -L/nobackupp10/scvasque/daily_builds/intel/esmf/lib/libg/ \
	Linux.intel.64.mpiuni.default  -L/nasa/intel/fce/10.1.021/lib/
ESMF_CXXLINKRPATHS: -Wl,-rpath,/nobackupp10/scvasque/daily_builds/intel/esmf/ \
	lib/libg/Linux.intel.64.mpiuni.default -Wl,-rpath,/nasa/intel/fce/ \
	10.1.021/lib/
ESMF_CXXLINKLIBS:  -lifport -lifcoremt -limf -lsvml -lm -lipgo -lguide -lirc \
	-lpthread -lgcc_s -lgcc -lirc_s -ldl -lrt -ldl
ESMF_CXXESMFLINKLIBS: -lesmf  -lifport -lifcoremt -limf -lsvml -lm -lipgo \
	-lguide -lirc -lpthread -lgcc_s -lgcc -lirc_s -ldl -lrt -ldl


--------------------------------------------------------------
Compiling on Thu Oct 21 02:15:56 PDT 2010 on r75i0n8
Machine characteristics: Linux r75i0n8 2.6.16.60-0.68.1.20100916-nasa \
	#1 SMP Fri 
Sep 17 17:49:05 UTC 2010 x86_64 x86_64 x86_64 GNU/Linux
==============================================================

\end{verbatim}


\subsubsection{Building makefile targets}

The makefiles follow the GNU target standards where possible.
The most frequently used targets for building are listed below:
\begin{description}
\item[lib] build the ESMF libraries only (default)
%\item[all] build the libraries, unit and system tests, examples, and demos
\item[all] build the libraries, unit and system tests and examples
\item[doc] build the documentation (requires specific latex macros packages
and additional utilities; see Section \ref{sec:TechOver} for more details
on the requirements).  
\item[info] print out extensive system configuration information about what
           compilers, libraries, paths, flags, etc are being used
\item[clean] remove all files built for this platform/compiler/wordsize.
\item[clobber] remove all files built for all architectures
\item[install] install the ESMF library in a custom location
\end{description}


\subsubsection{Testing makefile targets}

To build and run the unit and system tests, type:
\begin{verbatim}
make check
\end{verbatim}
A summary report of success and failures will be printed out at the end.

See section \ref{ESMFRunSetting} on how to set up ESMF to be able to launch
the bundled test and example applications.

\noindent Other test-related targets are:
\begin{description}
\item[all\_tests] build and run all available tests and examples
\item[build\_all\_tests] build tests and examples; do not execute
\item[run\_all\_tests] run tests and examples without rebuilding; print a
summary of the results
\item[check\_all\_tests] print out the results summary without re-executing
\item[dust\_all\_tests] remove all test and example output files
\item[clean\_all\_tests] remove all test and example executables and output
files
\end{description}

For all the targets listed above, the string {\tt all\_tests} can be
replaced with one of the strings listed below to select a
specific type of test:
\begin{description}
\item[unit\_tests] unit tests exercise a single part of the system
\item[system\_tests] system tests combine functions across the system
\item[examples] examples contain code illustrating a single type of function
%\item[demos] demos are example applications showing the use of the system
\end{description}
For example, {\tt make build\_examples} recompiles the example programs but 
does not execute them. {\tt make dust\_unit\_tests} removes all
output files generated when executing the unit tests, but leaves the
executables. {\tt make clean\_system\_tests} removes all executables and files
associated with the system tests.

For the unit tests only, there is an additional environment variable
which affects how the tests are built:
\begin{description}
\item[ESMF\_TESTEXHAUSTIVE]
If this variable is set to {\tt ON} before compiling the unit tests,
longer and more exhaustive unit tests will be run.  Note that this is a
compile-time and not run-time option.
\end{description}


\subsubsection{Building and using bundled ESMF Command Line Tools}
\label{quickapps}

This section describes how the bundled ESMF command line tools can be built
and used from inside the ESMF source tree. Notice that this is sort of a quick
and dirty way of accessing the ESMF applications. It is supported as
convenience to those users interested in quickly gaining access to the bundled
ESMF command line tools, and do not mind the shortcomings of this approach.
Users interested in maximum portability should instead follow the instructions
provided in section \ref{sec:CLTs}.

To build the bundled ESMF command line tools, type:
\begin{verbatim}
make build_apps
\end{verbatim}
This will build the command line tools and place the executables under the 
{\tt \$ESMF\_DIR/apps} directory inside the ESMF source tree. The 
command line tools can be directly executed from within the
{\tt \$ESMF\_DIR/apps} directory following the system specific rules for
execution. The details will depend on whether ESMF was built with or without
MPI dependency. In the latter case the system specific rules for launching
parallel applications must be followed. System specific execution details on
this level are outside of ESMF's scope.

For most systems, the MPI version of the ESMF bundled command line tools can be 
executed by a command equivalent to:

\begin{verbatim}

mpirun -np X $(ESMF_DIR)/apps/..../<cli-name>

\end{verbatim}
 
where {\tt X} specifies the total number of PETs and {\tt cli-name} is the 
name of the specific ESMF command line tool to be executed. The {\tt ....} in
the path indicates the precise subdirectory structure under {\tt ./apps} which
follows the standard ESMF pattern also used for the {\tt ./tests} and
{\tt ./examples} subdirectories.

All bundled ESMF command line tool support the standard \verb+ '--help'+ command
line option that prints out information on its proper use. More detailed
instructions of the individual tools are available in the "Command Line Tools"
section of the {\it ESMF Reference Manual}.

