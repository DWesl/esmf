\label{MediatorCompMeta}
The Mediator Component metadata is implemented through ESMF\_Info. It can be accessed
using the JSON Pointer "/NUOPC/Instance/" prefix followed by the "Attribute name"
as per the table below. E.g. "Verbosity" is accessed using {\tt key="/NUOPC/Instance/Verbosity"}.

{\bf Note} that some of the Attribute names in the following table are longer than the table column width. In these cases the
Attribute name had to be broken into multiple lines. When that happens, a hyphen shows up to indicate the line break. The hyphen
is {\em not} part of the Attribute name!

\begin{longtable}{|p{.22\textwidth}|p{.5\textwidth}|p{.3\textwidth}|}
     \hline\hline
     {\bf Attribute name} & {\bf Definition} & {\bf Controlled vocabulary}\\
     \hline\hline
     {\tt Kind} & String value indicating component kind.& Mediator\\ \hline
     {\tt Verbosity} & String value, converted into an integer, and interpreted as a bit field. The lower 16 bits (0-15) are reserved to control verbosity of the generic component implementation. Higher bits are available for user level verbosity control. \newline
                       {\bf bit 0}: Intro/Extro of methods with indentation.\newline
                       {\bf bit 1}: Intro/Extro with memory info.\newline
                       {\bf bit 2}: Intro/Extro with garbage collection info.\newline
                       {\bf bit 3}: Intro/Extro with local VM info.\newline
                       {\bf bit 4}: Intro/Extro with ImportState info.\newline
                       {\bf bit 5}: Intro/Extro with ExportState info.\newline
                       {\bf bit 8}: Log Initialize phase with $>>>$, $<<<$, and currTime.\newline
                       {\bf bit 9}: Log Run phase with $>>>$, $<<<$, and currTime.\newline
                       {\bf bit 10}: Log Finalize phase with $>>>$, $<<<$, and currTime.\newline
                       {\bf bit 11}: Log info about data dependency during initialize resolution.\newline
                       {\bf bit 12}: Log run sequence execution.
                     & 0, 1, 2, ... \newline
                       "off" = 0 (default), \newline
                       "low": some verbosity, bits: 0, 8, 9, 10, 13\newline
                       "high": more verbosity, bits: 0, 4, 5, 8, 9, 10, 11, 12, 13, 14\newline
                       "max": all lower 16 bits\\ \hline
     {\tt Profiling} & String value, converted into an integer, and interpreted as a bit field. The lower 16 bits (0-15) are reserved to control profiling of the generic component implementation. Higher bits are available for user level profiling control. \newline
                       {\bf bit 0}: Top level profiling of {\em Initialize} phases.\newline
                       {\bf bit 1}: Specialization point profiling of {\em Initialize} phases.\newline
                       {\bf bit 2}: Additional profiling of internals of {\em Initialize} phases.\newline
                       {\bf bit 3}: Top level profiling of {\em Run} phases.\newline
                       {\bf bit 4}: Specialization point profiling of {\em Run} phases.\newline
                       {\bf bit 5}: Additional profiling of internals of {\em Run} phases.\newline
                       {\bf bit 6}: Top level profiling of {\em Finalize} phases.\newline
                       {\bf bit 7}: Specialization point profiling of {\em Finalize} phases.\newline
                       {\bf bit 8}: Additional profiling of internals of {\em Finalize} phases.\newline
                       {\bf bit 9}: Leading barrier for {\em Initialize} phases.\newline
                       {\bf bit 10}: Leading barrier for {\em Run} phases.\newline
                       {\bf bit 11}: Leading barrier for {\em Finalize} phases.\newline
                     & 0, 1, 2, ... \newline
                       "off" = 0 (default), \newline
                       "low": Top level profiling. \newline
                       "high": Top level, specialization point profiling, and additional profiling of internals. \newline
                       "max": All lower 16 bits set.\\ \hline
     {\tt Diagnostic} & String value, converted into an integer, and interpreted as a bit field. The lower 16 bits (0-15) are reserved to control diagnostic of the generic component implementation. Higher bits are available for user level diagnostic control. \newline
                       {\bf bit 0}: Dump fields of the importState on entering {\em Initialize} phases.\newline
                       {\bf bit 1}: Dump fields of the exportState on entering {\em Initialize} phases.\newline
                       {\bf bit 2}: Dump fields of the importState on exiting {\em Initialize} phases.\newline
                       {\bf bit 3}: Dump fields of the exportState on exiting {\em Initialize} phases.\newline
                       {\bf bit 4}: Dump fields of the importState on entering {\em Run} phases.\newline
                       {\bf bit 5}: Dump fields of the exportState on entering {\em Run} phases.\newline
                       {\bf bit 6}: Dump fields of the importState on exiting {\em Run} phases.\newline
                       {\bf bit 7}: Dump fields of the exportState on exiting {\em Run} phases.\newline
                       {\bf bit 8}: Dump fields of the importState on entering {\em Finalize} phases.\newline
                       {\bf bit 9}: Dump fields of the exportState on entering {\em Finalize} phases.\newline
                       {\bf bit 10}: Dump fields of the importState on exiting {\em Finalize} phases.\newline
                       {\bf bit 11}: Dump fields of the exportState on exiting {\em Finalize} phases.
                     & 0, 1, 2, ... \newline
                       "off" = 0 (default), \newline
                       "max": All lower 16 bits set.\\ \hline
     {\tt CompLabel} & String value holding the label under which the component was added to its parent driver.& {\em no restriction}\\ \hline
     {\tt InitializePhaseMap} & List of string values, mapping the logical NUOPC initialize phases, of a specific Initialize Phase Definition (IPD) version, to the actual ESMF initialize phase number under which the entry point is registered.& IPDvXXpY=Z, where XX = two-digit revision number, e.g. 01, Y = logical NUOPC phase number, Z = actual ESMF phase number, with Y, Z > 0 and Y, Z < 10 \\ \hline
     {\tt RunPhaseMap} & List of string values, mapping the logical NUOPC run phases to the actual ESMF run phase number under which the entry point is registered.& {\em label-string}=Z, where {\em label-string} can be chosen freely, and Z = actual ESMF phase number. \\ \hline
     {\tt FinalizePhaseMap} & List of string values, mapping the logical NUOPC finalize phases to the actual ESMF finalize phase number under which the entry point is registered.& {\em label-string}=Z, where {\em label-string} can be chosen freely, and Z = actual ESMF phase number. \\ \hline
     {\tt Internal\-InitializePhaseMap} & List of string values, mapping the logical NUOPC initialize phases, of a specific Initialize Phase Definition (IPD) version, to the actual ESMF initialize phase number under which the entry point is registered.& IPDvXXpY=Z, where XX = two-digit revision number, e.g. 01, Y = logical NUOPC phase number, Z = actual ESMF phase number, with Y, Z > 0 and Y, Z < 10 \\ \hline
     {\tt NestingGeneration} & Integer value enumerating nesting level.& 0, 1, 2, ...\\ \hline
     {\tt Nestling} & Integer value enumerating siblings within the same generation.& 0, 1, 2, ...\\ \hline
     {\tt Initialize\-DataComplete} & String value indicating whether all initialize data dependencies have been satisfied.& false, true\\ \hline
     {\tt Initialize\-DataProgress} & String value indicating whether progress is being made resolving initialize data dependencies.& false, true\\ \hline
     {\tt HierarchyProtocol} & String value specifying the hierarchy protocol.& "PushUpAllExportsAndUnsatisfiedImports" for field mirroring and connecting, "ConnectProvidedFields" to only connect provided fields (no mirroring), {\em All other values currently disable the hierarchy protocol.}\\ \hline
     \hline
\end{longtable}
