% $Id$
%
% Earth System Modeling Framework
% Copyright (c) 2002-2025, University Corporation for Atmospheric Research,
% Massachusetts Institute of Technology, Geophysical Fluid Dynamics
% Laboratory, University of Michigan, National Centers for Environmental
% Prediction, Los Alamos National Laboratory, Argonne National Laboratory,
% NASA Goddard Space Flight Center.
% Licensed under the University of Illinois-NCSA License.

\begin{enumerate}

\item{\bf No synchronization of object IDs at object create time - Unison Rule:}
Object IDs are used during the reconcile process to identify objects
which are unknown to some subset of the PETs in the currently running VM.
Object IDs are assigned in sequential order at object create time across the
context of the current VM without communication. This design was requested by
the user community during ESMF object design to reduce communication and
synchronization overhead when creating distributed ESMF objects.
As a consequence it is required to create distributed ESMF objects in
{\bf unison} across all PETs of the current VM in order to keep the ESMF object
identification in sync.

Violation of the unison rule will lead to undefined behavior when reconciling
a State that contains objects with inconsistent object IDs.

\item{\bf Info keys on top level State not reconciled without actual objects
present from the relevant sub-context.} One of the actions of the
{\tt ESMF\_StateReconcile()} method is to reconcile the Info keys of the
State object itself. The endresult is that the reconciled State has the
same Info {\em keys} on all of the PETs of the VM across which it was
reconciled -- albeit with potentially different values across PETs
(see the {\tt ESMF\_StateReconcile()} API doc for more details). An edge case
for which {\tt ESMF\_StateReconcile()} does {\bf not} provide Info key
reconcilation is when keys were added under a component executing on a subset
of PETs (compared to the reconciling VM), but no actual object
(Field, FieldBundle, Array, ArrayBundle, or nested State) was added under the
VM of that sub-context.

The situation of unreconciled Info keys across PETs for an ESMF State is not an
error condition per-se, however, it can lead to unexpected behavior in
downstream code. Specifically if such code expects to find consistent Info keys
across all PETs. If this is the case, care should be taken to ensure actual
objects are added to the top level State on the sub-context PETs where new Info
keys are added.

\end{enumerate}



